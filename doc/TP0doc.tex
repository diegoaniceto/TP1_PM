%Documentação do Trabalho Prático 0 de AEDSIII
%Édipo Fernandes Vieira de Oliveira - 2011054324
\documentclass[12pt, a4paper]{article}
\usepackage[utf8]{inputenc}
\usepackage[brazil]{babel} %idioma
\usepackage{ae} %caracteres especiais
\usepackage{amsmath, amsfonts, amssymb}
\usepackage{enumerate}
\usepackage[tmargin=2cm, bmargin=2cm, lmargin=2cm, rmargin=2cm]{geometry}
\usepackage{graphicx}
\usepackage[lined,algonl,ruled]{algorithm2e} %Para acrescentar algorithm
\begin{document}

 %capa

 \title{Trabalho Prático 1 \\ Rede Social de Pesquisadores \\}
 \author{Édipo Fernandes Vieira de Oliveira - 2011054324\\ Diego Henrique de Castro Aniceto - 2011054286\\Departamento de Ciência da Computação -- Universidade Federal de Minas Gerais\\}
  \maketitle

 %\newpage
 %sumario
  %use o section

\textbf{\textit \\Resumo: }
\textit{\\ Este relatório descreve a implementação da solução proposta para o o problema de manipulação de e armazenamento de dados de uma rede social de pesquisadores. Para que fosse possivel essa implemetação foi utilizado a linguagem Java de programação além de teorias de Orientação a Objeto e Modularização.\\
  O resultado obtido foi satisfatório, tanto em relação a solução do problema, quanto aos conceitos envolvidos.}
\section*{1. Introdução}

  Este trabalho tem como objetivo, introduzir os principais conceitos da Programação Orientada ao Objeto e modularização de código. Para que estes fossem exercitados foi proposto a solução de um problemas de manipulação e armazenamento de dados referentes a uma rede social de pesquisadores onde eles podem se relacionar através de artigos publicados, podem publicar seus artigos desenvidos, entre outras coisas \\\\
  A modelagem do problema gira em torno de três entidades principais, que são os Pesquisadores, os Veiculos de Comunicação e os Artigos, e a partir delas e dos arquivos de entradas disponibilizados, e a partir disso realizar os calculos solictidos pela especificação do trabalho, que são: o calculo de popularidade de cada Pesquisador, o fator de impacto de cada Veiculo de Comunicação, e a pontuação de cada Artigo.\\\\

  \begin{itemize}
  \item A seção 2 discute detalhes de implementação.
  \item A seção 3 traz os testes realizados para verificar a solução do trabalho, bem como a saída gerada
  \item A seção 4 apresenta uma breve conclusão sobre o trabalho.
  \item E por fim a seção 5 traz as referências bibliográficas.
\end{itemize}

%% Implementação %%
\section*{2. Implementação}

%% Testes %%
\section*{3. Testes}

%% Conclusão %%
\section*{4. Conclusão}


%% Referencias %%
\section*{5. Referências bibliográficas}
\end{document}
